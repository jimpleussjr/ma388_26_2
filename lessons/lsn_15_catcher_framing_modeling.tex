% Options for packages loaded elsewhere
\PassOptionsToPackage{unicode}{hyperref}
\PassOptionsToPackage{hyphens}{url}
\PassOptionsToPackage{dvipsnames,svgnames,x11names}{xcolor}
%
\documentclass[
]{article}

\usepackage{amsmath,amssymb}
\usepackage{iftex}
\ifPDFTeX
  \usepackage[T1]{fontenc}
  \usepackage[utf8]{inputenc}
  \usepackage{textcomp} % provide euro and other symbols
\else % if luatex or xetex
  \usepackage{unicode-math}
  \defaultfontfeatures{Scale=MatchLowercase}
  \defaultfontfeatures[\rmfamily]{Ligatures=TeX,Scale=1}
\fi
\usepackage{lmodern}
\ifPDFTeX\else  
    % xetex/luatex font selection
\fi
% Use upquote if available, for straight quotes in verbatim environments
\IfFileExists{upquote.sty}{\usepackage{upquote}}{}
\IfFileExists{microtype.sty}{% use microtype if available
  \usepackage[]{microtype}
  \UseMicrotypeSet[protrusion]{basicmath} % disable protrusion for tt fonts
}{}
\makeatletter
\@ifundefined{KOMAClassName}{% if non-KOMA class
  \IfFileExists{parskip.sty}{%
    \usepackage{parskip}
  }{% else
    \setlength{\parindent}{0pt}
    \setlength{\parskip}{6pt plus 2pt minus 1pt}}
}{% if KOMA class
  \KOMAoptions{parskip=half}}
\makeatother
\usepackage{xcolor}
\setlength{\emergencystretch}{3em} % prevent overfull lines
\setcounter{secnumdepth}{-\maxdimen} % remove section numbering
% Make \paragraph and \subparagraph free-standing
\makeatletter
\ifx\paragraph\undefined\else
  \let\oldparagraph\paragraph
  \renewcommand{\paragraph}{
    \@ifstar
      \xxxParagraphStar
      \xxxParagraphNoStar
  }
  \newcommand{\xxxParagraphStar}[1]{\oldparagraph*{#1}\mbox{}}
  \newcommand{\xxxParagraphNoStar}[1]{\oldparagraph{#1}\mbox{}}
\fi
\ifx\subparagraph\undefined\else
  \let\oldsubparagraph\subparagraph
  \renewcommand{\subparagraph}{
    \@ifstar
      \xxxSubParagraphStar
      \xxxSubParagraphNoStar
  }
  \newcommand{\xxxSubParagraphStar}[1]{\oldsubparagraph*{#1}\mbox{}}
  \newcommand{\xxxSubParagraphNoStar}[1]{\oldsubparagraph{#1}\mbox{}}
\fi
\makeatother

\usepackage{color}
\usepackage{fancyvrb}
\newcommand{\VerbBar}{|}
\newcommand{\VERB}{\Verb[commandchars=\\\{\}]}
\DefineVerbatimEnvironment{Highlighting}{Verbatim}{commandchars=\\\{\}}
% Add ',fontsize=\small' for more characters per line
\usepackage{framed}
\definecolor{shadecolor}{RGB}{241,243,245}
\newenvironment{Shaded}{\begin{snugshade}}{\end{snugshade}}
\newcommand{\AlertTok}[1]{\textcolor[rgb]{0.68,0.00,0.00}{#1}}
\newcommand{\AnnotationTok}[1]{\textcolor[rgb]{0.37,0.37,0.37}{#1}}
\newcommand{\AttributeTok}[1]{\textcolor[rgb]{0.40,0.45,0.13}{#1}}
\newcommand{\BaseNTok}[1]{\textcolor[rgb]{0.68,0.00,0.00}{#1}}
\newcommand{\BuiltInTok}[1]{\textcolor[rgb]{0.00,0.23,0.31}{#1}}
\newcommand{\CharTok}[1]{\textcolor[rgb]{0.13,0.47,0.30}{#1}}
\newcommand{\CommentTok}[1]{\textcolor[rgb]{0.37,0.37,0.37}{#1}}
\newcommand{\CommentVarTok}[1]{\textcolor[rgb]{0.37,0.37,0.37}{\textit{#1}}}
\newcommand{\ConstantTok}[1]{\textcolor[rgb]{0.56,0.35,0.01}{#1}}
\newcommand{\ControlFlowTok}[1]{\textcolor[rgb]{0.00,0.23,0.31}{\textbf{#1}}}
\newcommand{\DataTypeTok}[1]{\textcolor[rgb]{0.68,0.00,0.00}{#1}}
\newcommand{\DecValTok}[1]{\textcolor[rgb]{0.68,0.00,0.00}{#1}}
\newcommand{\DocumentationTok}[1]{\textcolor[rgb]{0.37,0.37,0.37}{\textit{#1}}}
\newcommand{\ErrorTok}[1]{\textcolor[rgb]{0.68,0.00,0.00}{#1}}
\newcommand{\ExtensionTok}[1]{\textcolor[rgb]{0.00,0.23,0.31}{#1}}
\newcommand{\FloatTok}[1]{\textcolor[rgb]{0.68,0.00,0.00}{#1}}
\newcommand{\FunctionTok}[1]{\textcolor[rgb]{0.28,0.35,0.67}{#1}}
\newcommand{\ImportTok}[1]{\textcolor[rgb]{0.00,0.46,0.62}{#1}}
\newcommand{\InformationTok}[1]{\textcolor[rgb]{0.37,0.37,0.37}{#1}}
\newcommand{\KeywordTok}[1]{\textcolor[rgb]{0.00,0.23,0.31}{\textbf{#1}}}
\newcommand{\NormalTok}[1]{\textcolor[rgb]{0.00,0.23,0.31}{#1}}
\newcommand{\OperatorTok}[1]{\textcolor[rgb]{0.37,0.37,0.37}{#1}}
\newcommand{\OtherTok}[1]{\textcolor[rgb]{0.00,0.23,0.31}{#1}}
\newcommand{\PreprocessorTok}[1]{\textcolor[rgb]{0.68,0.00,0.00}{#1}}
\newcommand{\RegionMarkerTok}[1]{\textcolor[rgb]{0.00,0.23,0.31}{#1}}
\newcommand{\SpecialCharTok}[1]{\textcolor[rgb]{0.37,0.37,0.37}{#1}}
\newcommand{\SpecialStringTok}[1]{\textcolor[rgb]{0.13,0.47,0.30}{#1}}
\newcommand{\StringTok}[1]{\textcolor[rgb]{0.13,0.47,0.30}{#1}}
\newcommand{\VariableTok}[1]{\textcolor[rgb]{0.07,0.07,0.07}{#1}}
\newcommand{\VerbatimStringTok}[1]{\textcolor[rgb]{0.13,0.47,0.30}{#1}}
\newcommand{\WarningTok}[1]{\textcolor[rgb]{0.37,0.37,0.37}{\textit{#1}}}

\providecommand{\tightlist}{%
  \setlength{\itemsep}{0pt}\setlength{\parskip}{0pt}}\usepackage{longtable,booktabs,array}
\usepackage{calc} % for calculating minipage widths
% Correct order of tables after \paragraph or \subparagraph
\usepackage{etoolbox}
\makeatletter
\patchcmd\longtable{\par}{\if@noskipsec\mbox{}\fi\par}{}{}
\makeatother
% Allow footnotes in longtable head/foot
\IfFileExists{footnotehyper.sty}{\usepackage{footnotehyper}}{\usepackage{footnote}}
\makesavenoteenv{longtable}
\usepackage{graphicx}
\makeatletter
\newsavebox\pandoc@box
\newcommand*\pandocbounded[1]{% scales image to fit in text height/width
  \sbox\pandoc@box{#1}%
  \Gscale@div\@tempa{\textheight}{\dimexpr\ht\pandoc@box+\dp\pandoc@box\relax}%
  \Gscale@div\@tempb{\linewidth}{\wd\pandoc@box}%
  \ifdim\@tempb\p@<\@tempa\p@\let\@tempa\@tempb\fi% select the smaller of both
  \ifdim\@tempa\p@<\p@\scalebox{\@tempa}{\usebox\pandoc@box}%
  \else\usebox{\pandoc@box}%
  \fi%
}
% Set default figure placement to htbp
\def\fps@figure{htbp}
\makeatother

\makeatletter
\@ifpackageloaded{caption}{}{\usepackage{caption}}
\AtBeginDocument{%
\ifdefined\contentsname
  \renewcommand*\contentsname{Table of contents}
\else
  \newcommand\contentsname{Table of contents}
\fi
\ifdefined\listfigurename
  \renewcommand*\listfigurename{List of Figures}
\else
  \newcommand\listfigurename{List of Figures}
\fi
\ifdefined\listtablename
  \renewcommand*\listtablename{List of Tables}
\else
  \newcommand\listtablename{List of Tables}
\fi
\ifdefined\figurename
  \renewcommand*\figurename{Figure}
\else
  \newcommand\figurename{Figure}
\fi
\ifdefined\tablename
  \renewcommand*\tablename{Table}
\else
  \newcommand\tablename{Table}
\fi
}
\@ifpackageloaded{float}{}{\usepackage{float}}
\floatstyle{ruled}
\@ifundefined{c@chapter}{\newfloat{codelisting}{h}{lop}}{\newfloat{codelisting}{h}{lop}[chapter]}
\floatname{codelisting}{Listing}
\newcommand*\listoflistings{\listof{codelisting}{List of Listings}}
\makeatother
\makeatletter
\makeatother
\makeatletter
\@ifpackageloaded{caption}{}{\usepackage{caption}}
\@ifpackageloaded{subcaption}{}{\usepackage{subcaption}}
\makeatother

\usepackage{bookmark}

\IfFileExists{xurl.sty}{\usepackage{xurl}}{} % add URL line breaks if available
\urlstyle{same} % disable monospaced font for URLs
\hypersetup{
  pdftitle={MA388 Sabermetrics: Lesson 15},
  pdfauthor={LTC Jim Pleuss},
  colorlinks=true,
  linkcolor={blue},
  filecolor={Maroon},
  citecolor={Blue},
  urlcolor={Blue},
  pdfcreator={LaTeX via pandoc}}


\title{MA388 Sabermetrics: Lesson 15}
\usepackage{etoolbox}
\makeatletter
\providecommand{\subtitle}[1]{% add subtitle to \maketitle
  \apptocmd{\@title}{\par {\large #1 \par}}{}{}
}
\makeatother
\subtitle{Catcher Framing Ability Modeling}
\author{LTC Jim Pleuss}
\date{}

\begin{document}
\maketitle


Last class, we began comparing called strikes for Buster Posey and
Yadier Molina. Today let's look at their results from June 2019.

\includegraphics[width=0.45\linewidth,height=\textheight,keepaspectratio]{molina.jpg}
\hspace{.5in}
\includegraphics[width=0.45\linewidth,height=\textheight,keepaspectratio]{posey.jpg}

Let's grab the data first. This should look familiar (see lesson 13).

\begin{Shaded}
\begin{Highlighting}[]
\FunctionTok{library}\NormalTok{(tidyverse)}
\FunctionTok{library}\NormalTok{(knitr)}
\FunctionTok{library}\NormalTok{(broom)}
\FunctionTok{library}\NormalTok{(baseballr)}
\end{Highlighting}
\end{Shaded}

\begin{Shaded}
\begin{Highlighting}[]
\CommentTok{\# Retrieve pitch level data from June 2019. (See previous lesson.)}
\CommentTok{\# If you\textquotesingle{}ve already collected the pitch data below, just load it.  Otherwise,}
\CommentTok{\# collect it from scratch.}

\ControlFlowTok{if}\NormalTok{(}\StringTok{"statcast\_may\_2019.rds"} \SpecialCharTok{\%in\%} \FunctionTok{dir}\NormalTok{(}\FunctionTok{getwd}\NormalTok{()))\{}
\NormalTok{  pitches }\OtherTok{\textless{}{-}} \FunctionTok{readRDS}\NormalTok{(}\StringTok{"statcast\_june\_2019.rds"}\NormalTok{)}
\NormalTok{\}}\ControlFlowTok{else}\NormalTok{\{}
  
  \CommentTok{\# Retrieve pitch{-}level data from June 2019.  }

\NormalTok{get\_statcast\_pitches }\OtherTok{\textless{}{-}} \ControlFlowTok{function}\NormalTok{(start\_day, end\_day, }\AttributeTok{chunk\_size =} \DecValTok{5}\NormalTok{) \{}

  \CommentTok{\# Coerce to Date}
\NormalTok{  start\_day }\OtherTok{\textless{}{-}} \FunctionTok{as.Date}\NormalTok{(start\_day)}
\NormalTok{  end\_day   }\OtherTok{\textless{}{-}} \FunctionTok{as.Date}\NormalTok{(end\_day)}

  \CommentTok{\# Create sequence of chunk start dates}
\NormalTok{  chunk\_starts }\OtherTok{\textless{}{-}} \FunctionTok{seq}\NormalTok{(start\_day, end\_day, }\AttributeTok{by =} \FunctionTok{paste}\NormalTok{(chunk\_size, }\StringTok{"days"}\NormalTok{))}

  \CommentTok{\# Build data}
\NormalTok{  pitch\_data }\OtherTok{\textless{}{-}} \FunctionTok{map\_dfr}\NormalTok{(chunk\_starts, }\ControlFlowTok{function}\NormalTok{(chunk\_start) \{}

\NormalTok{    chunk\_end }\OtherTok{\textless{}{-}} \FunctionTok{min}\NormalTok{(chunk\_start }\SpecialCharTok{+} \FunctionTok{days}\NormalTok{(chunk\_size }\SpecialCharTok{{-}} \DecValTok{1}\NormalTok{), end\_day)}
       \FunctionTok{statcast\_search}\NormalTok{(}
        \AttributeTok{start\_date =}\NormalTok{ chunk\_start,}
        \AttributeTok{end\_date   =}\NormalTok{ chunk\_end}
\NormalTok{      )}
\NormalTok{  \})}
  \FunctionTok{return}\NormalTok{(pitch\_data)}
\NormalTok{\}  }
  
  \CommentTok{\# If we want to limit to a certain number of pitches, we might take the head()}
  \CommentTok{\# or sample\_n() to get the number we want.}
\NormalTok{  pitches }\OtherTok{\textless{}{-}} \FunctionTok{get\_statcast\_pitches}\NormalTok{(}\StringTok{\textquotesingle{}2019{-}06{-}01\textquotesingle{}}\NormalTok{,}\StringTok{\textquotesingle{}2019{-}06{-}30\textquotesingle{}}\NormalTok{)}

  \FunctionTok{saveRDS}\NormalTok{(pitches, }\StringTok{"statcast\_june\_2019.rds"}\NormalTok{)}
\NormalTok{\}}

\NormalTok{pitches }\OtherTok{\textless{}{-}} \FunctionTok{readRDS}\NormalTok{(}\StringTok{"statcast\_june\_2019.rds"}\NormalTok{)}
\end{Highlighting}
\end{Shaded}

Now we can pull in the catcher name and display the strike and ball
counts for both catchers.

\begin{Shaded}
\begin{Highlighting}[]
\CommentTok{\# Add catcher\textquotesingle{}s name to the pitch data from the MLB master list.}

\NormalTok{mlbIDs }\OtherTok{\textless{}{-}} 
\NormalTok{  baseballr}\SpecialCharTok{::}\FunctionTok{chadwick\_player\_lu}\NormalTok{() }\SpecialCharTok{|\textgreater{}} 
  \FunctionTok{mutate}\NormalTok{(}\AttributeTok{mlb\_name =} \FunctionTok{paste}\NormalTok{(name\_first, name\_last),}
    \AttributeTok{mlb\_id =}\NormalTok{ key\_mlbam)}

\NormalTok{pitches }\OtherTok{\textless{}{-}}\NormalTok{ pitches }\SpecialCharTok{|\textgreater{}} 
  \FunctionTok{left\_join}\NormalTok{(}\FunctionTok{select}\NormalTok{(mlbIDs, mlb\_name, mlb\_id),}
            \AttributeTok{by =} \FunctionTok{c}\NormalTok{(}\StringTok{"fielder\_2"} \OtherTok{=} \StringTok{"mlb\_id"}\NormalTok{)) }\SpecialCharTok{|\textgreater{}} 
  \FunctionTok{rename}\NormalTok{(}\AttributeTok{catcher\_name =}\NormalTok{ mlb\_name)}

\CommentTok{\# Look only at pitches taken when Molina or Posey are catching.}
\NormalTok{pitches\_taken\_subset }\OtherTok{\textless{}{-}}\NormalTok{ pitches }\SpecialCharTok{|\textgreater{}} 
  \FunctionTok{filter}\NormalTok{(catcher\_name }\SpecialCharTok{\%in\%} \FunctionTok{c}\NormalTok{(}\StringTok{"Buster Posey"}\NormalTok{, }\StringTok{"Yadier Molina"}\NormalTok{),}
\NormalTok{         description }\SpecialCharTok{\%in\%} \FunctionTok{c}\NormalTok{(}\StringTok{"ball"}\NormalTok{, }\StringTok{"called\_strike"}\NormalTok{))}

\CommentTok{\# Form a 2x2 table.}
\NormalTok{pitches\_taken\_subset }\SpecialCharTok{|\textgreater{}} 
  \FunctionTok{count}\NormalTok{(catcher\_name, description) }\SpecialCharTok{|\textgreater{}} 
  \FunctionTok{pivot\_wider}\NormalTok{(}\AttributeTok{id\_cols =}\NormalTok{ description, }
              \AttributeTok{names\_from =}\NormalTok{ catcher\_name,}
              \AttributeTok{values\_from =}\NormalTok{ n) }\SpecialCharTok{|\textgreater{}} 
  \FunctionTok{kable}\NormalTok{(}\AttributeTok{caption =} \StringTok{"Results of taken pitches (June 2019)"}\NormalTok{)}
\end{Highlighting}
\end{Shaded}

\begin{longtable}[]{@{}lrr@{}}
\caption{Results of taken pitches (June 2019)}\tabularnewline
\toprule\noalign{}
description & Buster Posey & Yadier Molina \\
\midrule\noalign{}
\endfirsthead
\toprule\noalign{}
description & Buster Posey & Yadier Molina \\
\midrule\noalign{}
\endhead
\bottomrule\noalign{}
\endlastfoot
ball & 608 & 764 \\
called\_strike & 295 & 391 \\
\end{longtable}

\subsection{Logistic Regression}\label{logistic-regression}

We can use logistic regression to adjust for confounding variables.
Consider the results in Table 1 and the following logistic regression
model for the data. Let \(Y_i\) be a random variable for whether pitch
\(i\) was a called strike such that \(Y_i \sim \text{Bernoulli}(\pi_i)\)
and

\[\log\left(\frac{\pi_i}{1-\pi_i}\right) = \beta_0 + \beta_1 \text{Molina}_i\]

where \(\text{Molina}_i = 1\) if Yadier Molina was catcher and
\(\text{Molina}_i = 0\) if Buster Posey was catcher.

\textbf{Fit the model above and report the final model equation.}

\vspace{1in}

Based on this model, is there evidence Molina has more called strikes in
the long term?

\vspace{.5in}

A better analysis would also adjust for the location of the pitch. We
need a model for called strikes based on location. You might be tempted
to consider the following model:

\[\log\left(\frac{\pi_i}{1-\pi_i}\right) = \beta_0 + \beta_1 \text{plate\_x}_i + \beta_2  \text{plate\_z}_i\]

Is this an effective model for adjusting for location? Explain.

\vspace{1in}

Let's consider an alternative, specifically a smooth function of the
location. We can write the model like this:

\[\log\left(\frac{\pi_i}{1-\pi_i}\right) = \beta_0 + f(\text{plate\_x}_i,\text{plate\_z}_i)\]

Fit the model in R using the \texttt{mgcv} package (Section 7.4).

\begin{Shaded}
\begin{Highlighting}[]
\FunctionTok{library}\NormalTok{(mgcv)}

\NormalTok{strike\_mod }\OtherTok{\textless{}{-}} \FunctionTok{gam}\NormalTok{(description }\SpecialCharTok{==} \StringTok{"called\_strike"} \SpecialCharTok{\textasciitilde{}} \FunctionTok{s}\NormalTok{(plate\_x, plate\_z), }\AttributeTok{family =} \StringTok{"binomial"}\NormalTok{, }\AttributeTok{data =}\NormalTok{ pitches\_taken\_subset)}
\end{Highlighting}
\end{Shaded}

We want this model to produce strike predictions based on location.
Let's look at the predicted strike probability for a pitch at
\texttt{plate\_x\ =\ -1} and \texttt{plate\_z\ =\ 2.5} (Section 7.4.1).
This would place the pitch just outside the left edge of home plate
(from the pitcher's perspective) in the vertical middle of the strike
zone.

\begin{Shaded}
\begin{Highlighting}[]
\NormalTok{strike\_mod }\SpecialCharTok{|\textgreater{}} 
  \FunctionTok{augment}\NormalTok{(}\AttributeTok{type.predict =} \StringTok{"response"}\NormalTok{,}
          \AttributeTok{newdata =} \FunctionTok{data.frame}\NormalTok{(}\AttributeTok{plate\_x =} \SpecialCharTok{{-}}\DecValTok{1}\NormalTok{,}
                                \AttributeTok{plate\_z =} \FloatTok{2.5}\NormalTok{))}
\end{Highlighting}
\end{Shaded}

\begin{verbatim}
# A tibble: 1 x 4
  plate_x plate_z .fitted .se.fit
    <dbl>   <dbl>   <dbl>   <dbl>
1      -1     2.5   0.405  0.0875
\end{verbatim}

Now, let's look at the prediction surface. First, we create a grid of
points and then calculate the predictions at those points (Section 7.4.2
).

\begin{Shaded}
\begin{Highlighting}[]
\FunctionTok{library}\NormalTok{(modelr) }\CommentTok{\#data\_grid function}

\CommentTok{\# Create a grid.}
\NormalTok{grid }\OtherTok{\textless{}{-}}\NormalTok{ pitches\_taken\_subset }\SpecialCharTok{|\textgreater{}} 
  \FunctionTok{data\_grid}\NormalTok{(}\AttributeTok{plate\_x =} \FunctionTok{seq\_range}\NormalTok{(plate\_x, }\AttributeTok{n =} \DecValTok{100}\NormalTok{),}
            \AttributeTok{plate\_z =} \FunctionTok{seq\_range}\NormalTok{(plate\_z, }\AttributeTok{n =} \DecValTok{100}\NormalTok{))}

\NormalTok{grid }\SpecialCharTok{|\textgreater{}} \FunctionTok{head}\NormalTok{(}\DecValTok{5}\NormalTok{)}
\end{Highlighting}
\end{Shaded}

\begin{verbatim}
# A tibble: 5 x 2
  plate_x plate_z
    <dbl>   <dbl>
1   -3.16   -1.56
2   -3.16   -1.46
3   -3.16   -1.36
4   -3.16   -1.26
5   -3.16   -1.16
\end{verbatim}

\begin{Shaded}
\begin{Highlighting}[]
\CommentTok{\# Calculate predicted probabilities for strikes on the grid.}
\NormalTok{grid }\OtherTok{\textless{}{-}}\NormalTok{ strike\_mod }\SpecialCharTok{|\textgreater{}} 
  \FunctionTok{augment}\NormalTok{(}\AttributeTok{type.predict =} \StringTok{"response"}\NormalTok{,}
          \AttributeTok{newdata =}\NormalTok{ grid)}

\NormalTok{grid }\SpecialCharTok{|\textgreater{}} \FunctionTok{head}\NormalTok{(}\DecValTok{5}\NormalTok{)}
\end{Highlighting}
\end{Shaded}

\begin{verbatim}
# A tibble: 5 x 4
  plate_x plate_z .fitted .se.fit
    <dbl>   <dbl>   <dbl>   <dbl>
1   -3.16   -1.56 0.00275   0.761
2   -3.16   -1.46 0.00260   0.703
3   -3.16   -1.36 0.00240   0.632
4   -3.16   -1.26 0.00215   0.552
5   -3.16   -1.16 0.00187   0.468
\end{verbatim}

Plot the results on the strike zone.

\begin{Shaded}
\begin{Highlighting}[]
\NormalTok{plate\_width }\OtherTok{\textless{}{-}} \DecValTok{17} \SpecialCharTok{+} \DecValTok{2} \SpecialCharTok{*}\NormalTok{ (}\DecValTok{9}\SpecialCharTok{/}\NormalTok{pi)}
\NormalTok{k\_zone\_plot }\OtherTok{\textless{}{-}} \FunctionTok{ggplot}\NormalTok{(}\ConstantTok{NULL}\NormalTok{, }\FunctionTok{aes}\NormalTok{(}\AttributeTok{x =}\NormalTok{ plate\_x, }\AttributeTok{y =}\NormalTok{ plate\_z)) }\SpecialCharTok{+}  
  \FunctionTok{geom\_rect}\NormalTok{(}\AttributeTok{xmin =} \SpecialCharTok{{-}}\NormalTok{(plate\_width}\SpecialCharTok{/}\DecValTok{2}\NormalTok{)}\SpecialCharTok{/}\DecValTok{12}\NormalTok{,}
            \AttributeTok{xmax =}\NormalTok{ (plate\_width}\SpecialCharTok{/}\DecValTok{2}\NormalTok{)}\SpecialCharTok{/}\DecValTok{12}\NormalTok{,}
            \AttributeTok{ymin =} \FloatTok{1.5}\NormalTok{,}
            \AttributeTok{ymax =} \FloatTok{3.6}\NormalTok{, }
            \AttributeTok{color =} \StringTok{"blue"}\NormalTok{,}
            \AttributeTok{alpha =} \DecValTok{0}\NormalTok{) }\SpecialCharTok{+}
  \FunctionTok{coord\_equal}\NormalTok{() }\SpecialCharTok{+}
  \FunctionTok{scale\_x\_continuous}\NormalTok{(}\StringTok{"Horizontal location (ft.)"}\NormalTok{, }\AttributeTok{limits =} \FunctionTok{c}\NormalTok{(}\SpecialCharTok{{-}}\DecValTok{2}\NormalTok{,}\DecValTok{2}\NormalTok{)) }\SpecialCharTok{+}
  \FunctionTok{scale\_y\_continuous}\NormalTok{(}\StringTok{"Vertical location (ft.)"}\NormalTok{, }\AttributeTok{limits =} \FunctionTok{c}\NormalTok{(}\DecValTok{0}\NormalTok{,}\DecValTok{5}\NormalTok{))}

\NormalTok{k\_zone\_plot }\SpecialCharTok{\%+\%} 
\NormalTok{  grid }\SpecialCharTok{+} 
  \FunctionTok{geom\_tile}\NormalTok{(}\FunctionTok{aes}\NormalTok{(}\AttributeTok{fill =}\NormalTok{ .fitted), }\AttributeTok{alpha =} \FloatTok{0.7}\NormalTok{) }\SpecialCharTok{+}
  \FunctionTok{scale\_fill\_gradient}\NormalTok{(}\AttributeTok{low =} \StringTok{"gray92"}\NormalTok{, }\AttributeTok{high =} \StringTok{"blue"}\NormalTok{) }\SpecialCharTok{+}
  \FunctionTok{labs}\NormalTok{(}\AttributeTok{fill =} \StringTok{"Called Strike Prob"}\NormalTok{)}
\end{Highlighting}
\end{Shaded}

\pandocbounded{\includegraphics[keepaspectratio]{lsn_15_catcher_framing_modeling_files/figure-pdf/unnamed-chunk-8-1.pdf}}

Now, we have a model for called strikes and pitch location. We can use
this model adjust for pitch location in our catcher model:

\[\log\left(\frac{\pi_i}{1-\pi_i}\right) = \beta_0 + \beta_1 \text{Molina}_i +  f(\text{plate\_x}_i,\text{plate\_z}_i)\]

Interpret \(\beta_1\) in this model.

\vspace{0.5in}

\begin{Shaded}
\begin{Highlighting}[]
\NormalTok{strike\_mod\_molina }\OtherTok{\textless{}{-}} \FunctionTok{gam}\NormalTok{(description }\SpecialCharTok{==} \StringTok{"called\_strike"} \SpecialCharTok{\textasciitilde{}} \FunctionTok{s}\NormalTok{(plate\_x, plate\_z) }\SpecialCharTok{+}\NormalTok{ catcher\_name,}
                  \AttributeTok{family =} \StringTok{"binomial"}\NormalTok{, }\AttributeTok{data =}\NormalTok{ pitches\_taken\_subset)}
\NormalTok{strike\_mod\_molina }\SpecialCharTok{|\textgreater{}} 
  \FunctionTok{summary}\NormalTok{()}
\end{Highlighting}
\end{Shaded}

\begin{verbatim}

Family: binomial 
Link function: logit 

Formula:
description == "called_strike" ~ s(plate_x, plate_z) + catcher_name

Parametric coefficients:
                          Estimate Std. Error z value Pr(>|z|)   
(Intercept)               -13.5836     5.1579  -2.634  0.00845 **
catcher_nameYadier Molina   0.2329     0.1981   1.175  0.23986   
---
Signif. codes:  0 '***' 0.001 '**' 0.01 '*' 0.05 '.' 0.1 ' ' 1

Approximate significance of smooth terms:
                     edf Ref.df Chi.sq p-value    
s(plate_x,plate_z) 25.04  26.13  276.2  <2e-16 ***
---
Signif. codes:  0 '***' 0.001 '**' 0.01 '*' 0.05 '.' 0.1 ' ' 1

R-sq.(adj) =  0.757   Deviance explained = 73.6%
UBRE = -0.63719  Scale est. = 1         n = 2057
\end{verbatim}

Based on this analysis, is there evidence that Molina has a higher
called strike probability after adjusting for pitch location? Explain.

\vspace{1in}

Right now we're looking at just two catchers in a binary categorical
sense. How might we extend this to many catchers at the same time?

\vspace{1in}




\end{document}
