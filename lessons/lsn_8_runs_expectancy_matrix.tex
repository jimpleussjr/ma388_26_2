% Options for packages loaded elsewhere
\PassOptionsToPackage{unicode}{hyperref}
\PassOptionsToPackage{hyphens}{url}
\PassOptionsToPackage{dvipsnames,svgnames,x11names}{xcolor}
%
\documentclass[
]{article}

\usepackage{amsmath,amssymb}
\usepackage{iftex}
\ifPDFTeX
  \usepackage[T1]{fontenc}
  \usepackage[utf8]{inputenc}
  \usepackage{textcomp} % provide euro and other symbols
\else % if luatex or xetex
  \usepackage{unicode-math}
  \defaultfontfeatures{Scale=MatchLowercase}
  \defaultfontfeatures[\rmfamily]{Ligatures=TeX,Scale=1}
\fi
\usepackage{lmodern}
\ifPDFTeX\else  
    % xetex/luatex font selection
\fi
% Use upquote if available, for straight quotes in verbatim environments
\IfFileExists{upquote.sty}{\usepackage{upquote}}{}
\IfFileExists{microtype.sty}{% use microtype if available
  \usepackage[]{microtype}
  \UseMicrotypeSet[protrusion]{basicmath} % disable protrusion for tt fonts
}{}
\makeatletter
\@ifundefined{KOMAClassName}{% if non-KOMA class
  \IfFileExists{parskip.sty}{%
    \usepackage{parskip}
  }{% else
    \setlength{\parindent}{0pt}
    \setlength{\parskip}{6pt plus 2pt minus 1pt}}
}{% if KOMA class
  \KOMAoptions{parskip=half}}
\makeatother
\usepackage{xcolor}
\setlength{\emergencystretch}{3em} % prevent overfull lines
\setcounter{secnumdepth}{-\maxdimen} % remove section numbering
% Make \paragraph and \subparagraph free-standing
\makeatletter
\ifx\paragraph\undefined\else
  \let\oldparagraph\paragraph
  \renewcommand{\paragraph}{
    \@ifstar
      \xxxParagraphStar
      \xxxParagraphNoStar
  }
  \newcommand{\xxxParagraphStar}[1]{\oldparagraph*{#1}\mbox{}}
  \newcommand{\xxxParagraphNoStar}[1]{\oldparagraph{#1}\mbox{}}
\fi
\ifx\subparagraph\undefined\else
  \let\oldsubparagraph\subparagraph
  \renewcommand{\subparagraph}{
    \@ifstar
      \xxxSubParagraphStar
      \xxxSubParagraphNoStar
  }
  \newcommand{\xxxSubParagraphStar}[1]{\oldsubparagraph*{#1}\mbox{}}
  \newcommand{\xxxSubParagraphNoStar}[1]{\oldsubparagraph{#1}\mbox{}}
\fi
\makeatother

\usepackage{color}
\usepackage{fancyvrb}
\newcommand{\VerbBar}{|}
\newcommand{\VERB}{\Verb[commandchars=\\\{\}]}
\DefineVerbatimEnvironment{Highlighting}{Verbatim}{commandchars=\\\{\}}
% Add ',fontsize=\small' for more characters per line
\usepackage{framed}
\definecolor{shadecolor}{RGB}{241,243,245}
\newenvironment{Shaded}{\begin{snugshade}}{\end{snugshade}}
\newcommand{\AlertTok}[1]{\textcolor[rgb]{0.68,0.00,0.00}{#1}}
\newcommand{\AnnotationTok}[1]{\textcolor[rgb]{0.37,0.37,0.37}{#1}}
\newcommand{\AttributeTok}[1]{\textcolor[rgb]{0.40,0.45,0.13}{#1}}
\newcommand{\BaseNTok}[1]{\textcolor[rgb]{0.68,0.00,0.00}{#1}}
\newcommand{\BuiltInTok}[1]{\textcolor[rgb]{0.00,0.23,0.31}{#1}}
\newcommand{\CharTok}[1]{\textcolor[rgb]{0.13,0.47,0.30}{#1}}
\newcommand{\CommentTok}[1]{\textcolor[rgb]{0.37,0.37,0.37}{#1}}
\newcommand{\CommentVarTok}[1]{\textcolor[rgb]{0.37,0.37,0.37}{\textit{#1}}}
\newcommand{\ConstantTok}[1]{\textcolor[rgb]{0.56,0.35,0.01}{#1}}
\newcommand{\ControlFlowTok}[1]{\textcolor[rgb]{0.00,0.23,0.31}{\textbf{#1}}}
\newcommand{\DataTypeTok}[1]{\textcolor[rgb]{0.68,0.00,0.00}{#1}}
\newcommand{\DecValTok}[1]{\textcolor[rgb]{0.68,0.00,0.00}{#1}}
\newcommand{\DocumentationTok}[1]{\textcolor[rgb]{0.37,0.37,0.37}{\textit{#1}}}
\newcommand{\ErrorTok}[1]{\textcolor[rgb]{0.68,0.00,0.00}{#1}}
\newcommand{\ExtensionTok}[1]{\textcolor[rgb]{0.00,0.23,0.31}{#1}}
\newcommand{\FloatTok}[1]{\textcolor[rgb]{0.68,0.00,0.00}{#1}}
\newcommand{\FunctionTok}[1]{\textcolor[rgb]{0.28,0.35,0.67}{#1}}
\newcommand{\ImportTok}[1]{\textcolor[rgb]{0.00,0.46,0.62}{#1}}
\newcommand{\InformationTok}[1]{\textcolor[rgb]{0.37,0.37,0.37}{#1}}
\newcommand{\KeywordTok}[1]{\textcolor[rgb]{0.00,0.23,0.31}{\textbf{#1}}}
\newcommand{\NormalTok}[1]{\textcolor[rgb]{0.00,0.23,0.31}{#1}}
\newcommand{\OperatorTok}[1]{\textcolor[rgb]{0.37,0.37,0.37}{#1}}
\newcommand{\OtherTok}[1]{\textcolor[rgb]{0.00,0.23,0.31}{#1}}
\newcommand{\PreprocessorTok}[1]{\textcolor[rgb]{0.68,0.00,0.00}{#1}}
\newcommand{\RegionMarkerTok}[1]{\textcolor[rgb]{0.00,0.23,0.31}{#1}}
\newcommand{\SpecialCharTok}[1]{\textcolor[rgb]{0.37,0.37,0.37}{#1}}
\newcommand{\SpecialStringTok}[1]{\textcolor[rgb]{0.13,0.47,0.30}{#1}}
\newcommand{\StringTok}[1]{\textcolor[rgb]{0.13,0.47,0.30}{#1}}
\newcommand{\VariableTok}[1]{\textcolor[rgb]{0.07,0.07,0.07}{#1}}
\newcommand{\VerbatimStringTok}[1]{\textcolor[rgb]{0.13,0.47,0.30}{#1}}
\newcommand{\WarningTok}[1]{\textcolor[rgb]{0.37,0.37,0.37}{\textit{#1}}}

\providecommand{\tightlist}{%
  \setlength{\itemsep}{0pt}\setlength{\parskip}{0pt}}\usepackage{longtable,booktabs,array}
\usepackage{calc} % for calculating minipage widths
% Correct order of tables after \paragraph or \subparagraph
\usepackage{etoolbox}
\makeatletter
\patchcmd\longtable{\par}{\if@noskipsec\mbox{}\fi\par}{}{}
\makeatother
% Allow footnotes in longtable head/foot
\IfFileExists{footnotehyper.sty}{\usepackage{footnotehyper}}{\usepackage{footnote}}
\makesavenoteenv{longtable}
\usepackage{graphicx}
\makeatletter
\newsavebox\pandoc@box
\newcommand*\pandocbounded[1]{% scales image to fit in text height/width
  \sbox\pandoc@box{#1}%
  \Gscale@div\@tempa{\textheight}{\dimexpr\ht\pandoc@box+\dp\pandoc@box\relax}%
  \Gscale@div\@tempb{\linewidth}{\wd\pandoc@box}%
  \ifdim\@tempb\p@<\@tempa\p@\let\@tempa\@tempb\fi% select the smaller of both
  \ifdim\@tempa\p@<\p@\scalebox{\@tempa}{\usebox\pandoc@box}%
  \else\usebox{\pandoc@box}%
  \fi%
}
% Set default figure placement to htbp
\def\fps@figure{htbp}
\makeatother

\makeatletter
\@ifpackageloaded{caption}{}{\usepackage{caption}}
\AtBeginDocument{%
\ifdefined\contentsname
  \renewcommand*\contentsname{Table of contents}
\else
  \newcommand\contentsname{Table of contents}
\fi
\ifdefined\listfigurename
  \renewcommand*\listfigurename{List of Figures}
\else
  \newcommand\listfigurename{List of Figures}
\fi
\ifdefined\listtablename
  \renewcommand*\listtablename{List of Tables}
\else
  \newcommand\listtablename{List of Tables}
\fi
\ifdefined\figurename
  \renewcommand*\figurename{Figure}
\else
  \newcommand\figurename{Figure}
\fi
\ifdefined\tablename
  \renewcommand*\tablename{Table}
\else
  \newcommand\tablename{Table}
\fi
}
\@ifpackageloaded{float}{}{\usepackage{float}}
\floatstyle{ruled}
\@ifundefined{c@chapter}{\newfloat{codelisting}{h}{lop}}{\newfloat{codelisting}{h}{lop}[chapter]}
\floatname{codelisting}{Listing}
\newcommand*\listoflistings{\listof{codelisting}{List of Listings}}
\makeatother
\makeatletter
\makeatother
\makeatletter
\@ifpackageloaded{caption}{}{\usepackage{caption}}
\@ifpackageloaded{subcaption}{}{\usepackage{subcaption}}
\makeatother

\usepackage{bookmark}

\IfFileExists{xurl.sty}{\usepackage{xurl}}{} % add URL line breaks if available
\urlstyle{same} % disable monospaced font for URLs
\hypersetup{
  pdftitle={MA388 Sabermetrics: Lesson 8},
  pdfauthor={LTC Jim Pleuss},
  colorlinks=true,
  linkcolor={blue},
  filecolor={Maroon},
  citecolor={Blue},
  urlcolor={Blue},
  pdfcreator={LaTeX via pandoc}}


\title{MA388 Sabermetrics: Lesson 8}
\usepackage{etoolbox}
\makeatletter
\providecommand{\subtitle}[1]{% add subtitle to \maketitle
  \apptocmd{\@title}{\par {\large #1 \par}}{}{}
}
\makeatother
\subtitle{Value of Plays - Run Expectancy Matrix}
\author{LTC Jim Pleuss}
\date{}

\begin{document}
\maketitle


\begin{Shaded}
\begin{Highlighting}[]
\FunctionTok{library}\NormalTok{(tidyverse)}
\FunctionTok{library}\NormalTok{(Lahman)}
\FunctionTok{library}\NormalTok{(knitr)}
\FunctionTok{library}\NormalTok{(ggrepel)}
\FunctionTok{library}\NormalTok{(broom)}
\end{Highlighting}
\end{Shaded}

\subsection{Review}\label{review}

Last lesson, we discussed the following three models:

\begin{align}
Wpct &= \beta_0 + \beta_1 RD + \epsilon \\
Wpct &= \frac{R^2}{R^2 + RA^2} + \epsilon \\
Wpct &= \frac{R^k}{R^k + RA^k} + \epsilon
\end{align}

\begin{itemize}
\item
  Determine the value of \(k\) in the third model for the 2018 season.
\item
  Calculate the predicted wins for each team in 2018 and plot the
  residuals vs.~the predicted values.
\end{itemize}

\begin{figure}[H]

{\centering \pandocbounded{\includegraphics[keepaspectratio]{lsn_8_runs_expectancy_matrix_files/figure-pdf/unnamed-chunk-2-1.pdf}}

}

\caption{Pythagorean predictions, k = 1.75 (2018)}

\end{figure}%

\newpage

\subsection{Value of Plays}\label{value-of-plays}

Using the 10-runs-per-win rule of thumb (or more precise estimates using
the other models), we now have a nice way of converting runs to wins.
This leads to our next obvious question -- how do we score runs? Simply
put, players make plays, and sometimes those plays lead to runs. It's
time to assess how many runs players and plays are worth.

\subsubsection{Run Expectancy Matrix}\label{run-expectancy-matrix}

*e first step is to calculate the \emph{run expectancy matrix}. The run
expectancy matrix tells us the average number of runs scored in the
remainder of an inning for each state (runner/out combination) of an
inning.

The state of the game consists of four digits. The first three digits
indicate the location of runners on base. The last digit indicates the
number of outs.

Examples:

\begin{itemize}
\item
  ``000 0'' - no runners on base; no outs
\item
  ``111 2'' - bases loaded; two outs
\item
  ``010 1'' - runner on second base; one out
\end{itemize}

\begin{figure}[H]

{\centering \pandocbounded{\includegraphics[keepaspectratio]{run expectancy.png}}

}

\caption{Run Expectancy Matrices (source:
\url{http://tangotiger.net/re24.html})}

\end{figure}%

\emph{How do we interpret the values in the matrix?}

\vspace{1in}

\emph{Do you see any trends over time?}

\vspace{1in}

We are looking at run expectancy matrices. There is another type of
matrix called a \href{http://tangotiger.net/re24.html}{run probability
matrix}. \emph{Explain how they differ.}

\vspace{1in}

\subsubsection{Run Expectancy Matrix from Retrosheet
Data}\label{run-expectancy-matrix-from-retrosheet-data}

Typically, we use Retrosheet play-by-play data to calculate the run
expectancy matrix. Below, we will calculate it for 2011 using the code
from our textbook.

\begin{Shaded}
\begin{Highlighting}[]
\CommentTok{\# Import Retrosheet play{-}by{-}play data from our textbook site.}

\NormalTok{site }\OtherTok{=} \StringTok{"https://raw.githubusercontent.com/maxtoki/baseball\_R/"}
\NormalTok{fields }\OtherTok{\textless{}{-}} \FunctionTok{read\_csv}\NormalTok{(}\AttributeTok{file =} \FunctionTok{paste}\NormalTok{(site, }\StringTok{"master/data/fields.csv"}\NormalTok{, }\AttributeTok{sep =}\StringTok{""}\NormalTok{))}
\NormalTok{retro2011 }\OtherTok{\textless{}{-}} \FunctionTok{read\_csv}\NormalTok{(}\AttributeTok{file =} \FunctionTok{paste}\NormalTok{(site, }\StringTok{"master/data/all2011.csv"}\NormalTok{, }\AttributeTok{sep =} \StringTok{""}\NormalTok{),}
                    \AttributeTok{col\_names =} \FunctionTok{pull}\NormalTok{(fields, Header),}
                    \AttributeTok{na =} \FunctionTok{character}\NormalTok{())}
\FunctionTok{colnames}\NormalTok{(retro2011) }\OtherTok{\textless{}{-}} \FunctionTok{tolower}\NormalTok{(}\FunctionTok{colnames}\NormalTok{(retro2011))}
\end{Highlighting}
\end{Shaded}

This data frame contains one line for every play in the 2011 season.
There are approximately 200,000 lines in this data set.

\begin{Shaded}
\begin{Highlighting}[]
\NormalTok{retro2011 }\SpecialCharTok{|\textgreater{}} 
  \FunctionTok{select}\NormalTok{(game\_id, away\_team\_id, inn\_ct, outs\_ct,}
\NormalTok{         bat\_id, pit\_id, event\_cd) }\SpecialCharTok{|\textgreater{}} 
  \FunctionTok{head}\NormalTok{(}\DecValTok{10}\NormalTok{) }\SpecialCharTok{|\textgreater{}} 
  \FunctionTok{kable}\NormalTok{()}
\end{Highlighting}
\end{Shaded}

\begin{longtable}[]{@{}
  >{\raggedright\arraybackslash}p{(\linewidth - 12\tabcolsep) * \real{0.1912}}
  >{\raggedright\arraybackslash}p{(\linewidth - 12\tabcolsep) * \real{0.1912}}
  >{\raggedleft\arraybackslash}p{(\linewidth - 12\tabcolsep) * \real{0.1029}}
  >{\raggedleft\arraybackslash}p{(\linewidth - 12\tabcolsep) * \real{0.1176}}
  >{\raggedright\arraybackslash}p{(\linewidth - 12\tabcolsep) * \real{0.1324}}
  >{\raggedright\arraybackslash}p{(\linewidth - 12\tabcolsep) * \real{0.1324}}
  >{\raggedleft\arraybackslash}p{(\linewidth - 12\tabcolsep) * \real{0.1324}}@{}}
\toprule\noalign{}
\begin{minipage}[b]{\linewidth}\raggedright
game\_id
\end{minipage} & \begin{minipage}[b]{\linewidth}\raggedright
away\_team\_id
\end{minipage} & \begin{minipage}[b]{\linewidth}\raggedleft
inn\_ct
\end{minipage} & \begin{minipage}[b]{\linewidth}\raggedleft
outs\_ct
\end{minipage} & \begin{minipage}[b]{\linewidth}\raggedright
bat\_id
\end{minipage} & \begin{minipage}[b]{\linewidth}\raggedright
pit\_id
\end{minipage} & \begin{minipage}[b]{\linewidth}\raggedleft
event\_cd
\end{minipage} \\
\midrule\noalign{}
\endhead
\bottomrule\noalign{}
\endlastfoot
ANA201104080 & TOR & 1 & 0 & davir003 & sante001 & 2 \\
ANA201104080 & TOR & 1 & 1 & nix-j001 & sante001 & 2 \\
ANA201104080 & TOR & 1 & 2 & bautj002 & sante001 & 3 \\
ANA201104080 & TOR & 1 & 0 & iztum001 & drabk001 & 14 \\
ANA201104080 & TOR & 1 & 0 & kendh001 & drabk001 & 3 \\
ANA201104080 & TOR & 1 & 1 & abreb001 & drabk001 & 6 \\
ANA201104080 & TOR & 1 & 2 & abreb001 & drabk001 & 14 \\
ANA201104080 & TOR & 1 & 2 & huntt001 & drabk001 & 20 \\
ANA201104080 & TOR & 1 & 2 & wellv001 & drabk001 & 2 \\
ANA201104080 & TOR & 2 & 0 & linda001 & sante001 & 2 \\
\end{longtable}

To understand what happened in this game, we'll need to know what all
the \href{https://www.retrosheet.org/eventfile.htm\%5D}{event codes} are
for various baseball outcomes described by the \texttt{event\_cd} field.

\begin{figure}[H]

{\centering \includegraphics[width=0.5\linewidth,height=\textheight,keepaspectratio]{eventcodes.png}

}

\caption{Event codes}

\end{figure}%

If we further consider the \texttt{event\_tx} and
\texttt{pitch\_seq\_tx} fields, we can more comprehensively reconstruct
each at-bat. Let's try to describe what happened in the first inning of
this game. We'll need to know the
\href{https://baseballtrainingworld.com/what-do-the-numbers-mean-in-a-double-play-with-examples/}{position
numbers} to interpret the event text strings.

\begin{figure}[H]

{\centering \pandocbounded{\includegraphics[keepaspectratio]{baseball_position_numbers.png}}

}

\caption{Baseball Position Numbers}

\end{figure}%

\newpage

\begin{Shaded}
\begin{Highlighting}[]
\NormalTok{retro2011 }\SpecialCharTok{|\textgreater{}} 
  \FunctionTok{select}\NormalTok{(inn\_ct, outs\_ct,}
\NormalTok{         bat\_id, pit\_id, pitch\_seq\_tx, event\_tx) }\SpecialCharTok{|\textgreater{}} 
  \FunctionTok{head}\NormalTok{(}\DecValTok{10}\NormalTok{) }\SpecialCharTok{|\textgreater{}} 
  \FunctionTok{kable}\NormalTok{()}
\end{Highlighting}
\end{Shaded}

\begin{longtable}[]{@{}
  >{\raggedleft\arraybackslash}p{(\linewidth - 10\tabcolsep) * \real{0.0946}}
  >{\raggedleft\arraybackslash}p{(\linewidth - 10\tabcolsep) * \real{0.1081}}
  >{\raggedright\arraybackslash}p{(\linewidth - 10\tabcolsep) * \real{0.1216}}
  >{\raggedright\arraybackslash}p{(\linewidth - 10\tabcolsep) * \real{0.1216}}
  >{\raggedright\arraybackslash}p{(\linewidth - 10\tabcolsep) * \real{0.1757}}
  >{\raggedright\arraybackslash}p{(\linewidth - 10\tabcolsep) * \real{0.3784}}@{}}
\toprule\noalign{}
\begin{minipage}[b]{\linewidth}\raggedleft
inn\_ct
\end{minipage} & \begin{minipage}[b]{\linewidth}\raggedleft
outs\_ct
\end{minipage} & \begin{minipage}[b]{\linewidth}\raggedright
bat\_id
\end{minipage} & \begin{minipage}[b]{\linewidth}\raggedright
pit\_id
\end{minipage} & \begin{minipage}[b]{\linewidth}\raggedright
pitch\_seq\_tx
\end{minipage} & \begin{minipage}[b]{\linewidth}\raggedright
event\_tx
\end{minipage} \\
\midrule\noalign{}
\endhead
\bottomrule\noalign{}
\endlastfoot
1 & 0 & davir003 & sante001 & FBSX & 9/F \\
1 & 1 & nix-j001 & sante001 & X & 9/F \\
1 & 2 & bautj002 & sante001 & CBCS & K \\
1 & 0 & iztum001 & drabk001 & CBBBB & W \\
1 & 0 & kendh001 & drabk001 & BCSBS & K \\
1 & 1 & abreb001 & drabk001 & CBB1\textgreater S & CS2(26) \\
1 & 2 & abreb001 & drabk001 & CBB1\textgreater S.FBFB & W \\
1 & 2 & huntt001 & drabk001 & CCX & S9/L.1-H(E9/TH)(UR)(NR);B-3 \\
1 & 2 & wellv001 & drabk001 & BX & 63/G \\
2 & 0 & linda001 & sante001 & CBBFX & 3/L- \\
\end{longtable}

\newpage

\subsubsection{Calculate the Run Expectancy
Matrix}\label{calculate-the-run-expectancy-matrix}

The textbook provides the code below to generate the run expectancy
matrix.

\begin{Shaded}
\begin{Highlighting}[]
\NormalTok{retro2011 }\OtherTok{\textless{}{-}}\NormalTok{ retro2011 }\SpecialCharTok{|\textgreater{}} 
  \FunctionTok{mutate}\NormalTok{(}\AttributeTok{runs\_before =}\NormalTok{ away\_score\_ct }\SpecialCharTok{+}\NormalTok{ home\_score\_ct,}
         \AttributeTok{half\_inning =} \FunctionTok{paste}\NormalTok{(game\_id, inn\_ct, bat\_home\_id),}
         \AttributeTok{runs\_scored =} 
\NormalTok{           (bat\_dest\_id }\SpecialCharTok{\textgreater{}} \DecValTok{3}\NormalTok{) }\SpecialCharTok{+}\NormalTok{ (run1\_dest\_id }\SpecialCharTok{\textgreater{}} \DecValTok{3}\NormalTok{) }\SpecialCharTok{+} 
\NormalTok{           (run2\_dest\_id }\SpecialCharTok{\textgreater{}} \DecValTok{3}\NormalTok{) }\SpecialCharTok{+}\NormalTok{ (run3\_dest\_id }\SpecialCharTok{\textgreater{}} \DecValTok{3}\NormalTok{))}

\NormalTok{half\_innings }\OtherTok{\textless{}{-}}\NormalTok{ retro2011 }\SpecialCharTok{|\textgreater{}}
  \FunctionTok{group\_by}\NormalTok{(half\_inning) }\SpecialCharTok{|\textgreater{}}
  \FunctionTok{summarize}\NormalTok{(}\AttributeTok{outs\_inning =} \FunctionTok{sum}\NormalTok{(event\_outs\_ct),}
            \AttributeTok{runs\_inning =} \FunctionTok{sum}\NormalTok{(runs\_scored),}
            \AttributeTok{runs\_start =} \FunctionTok{first}\NormalTok{(runs\_before),}
            \AttributeTok{max\_runs =}\NormalTok{ runs\_inning }\SpecialCharTok{+}\NormalTok{ runs\_start)}

\NormalTok{retro2011 }\OtherTok{\textless{}{-}}\NormalTok{ retro2011 }\SpecialCharTok{|\textgreater{}}
  \FunctionTok{inner\_join}\NormalTok{(half\_innings, }\AttributeTok{by =} \StringTok{"half\_inning"}\NormalTok{) }\SpecialCharTok{|\textgreater{}}
  \FunctionTok{mutate}\NormalTok{(}\AttributeTok{runs\_roi =}\NormalTok{ max\_runs }\SpecialCharTok{{-}}\NormalTok{ runs\_before)}

\NormalTok{retro2011 }\OtherTok{\textless{}{-}}\NormalTok{ retro2011 }\SpecialCharTok{|\textgreater{}}
  \FunctionTok{mutate}\NormalTok{(}\AttributeTok{bases =} 
           \FunctionTok{paste}\NormalTok{(}\FunctionTok{ifelse}\NormalTok{(base1\_run\_id }\SpecialCharTok{\textgreater{}} \StringTok{\textquotesingle{}\textquotesingle{}}\NormalTok{,}\DecValTok{1}\NormalTok{,}\DecValTok{0}\NormalTok{),}
                 \FunctionTok{ifelse}\NormalTok{(base2\_run\_id }\SpecialCharTok{\textgreater{}} \StringTok{\textquotesingle{}\textquotesingle{}}\NormalTok{,}\DecValTok{1}\NormalTok{,}\DecValTok{0}\NormalTok{),}
                 \FunctionTok{ifelse}\NormalTok{(base3\_run\_id }\SpecialCharTok{\textgreater{}} \StringTok{\textquotesingle{}\textquotesingle{}}\NormalTok{,}\DecValTok{1}\NormalTok{,}\DecValTok{0}\NormalTok{), }\AttributeTok{sep =} \StringTok{""}\NormalTok{),}
         \AttributeTok{state =} \FunctionTok{paste}\NormalTok{(bases, outs\_ct))}

\NormalTok{retro2011 }\OtherTok{\textless{}{-}}\NormalTok{ retro2011 }\SpecialCharTok{|\textgreater{}}
  \FunctionTok{mutate}\NormalTok{(}\AttributeTok{is\_runner1 =}
           \FunctionTok{as.numeric}\NormalTok{(run1\_dest\_id}\SpecialCharTok{==}\DecValTok{1} \SpecialCharTok{|}\NormalTok{ bat\_dest\_id }\SpecialCharTok{==} \DecValTok{1}\NormalTok{),}
         \AttributeTok{is\_runner2 =} 
           \FunctionTok{as.numeric}\NormalTok{(run1\_dest\_id }\SpecialCharTok{==} \DecValTok{2} \SpecialCharTok{|}\NormalTok{ run2\_dest\_id }\SpecialCharTok{==} \DecValTok{2} \SpecialCharTok{|}
\NormalTok{                        bat\_dest\_id }\SpecialCharTok{==} \DecValTok{2}\NormalTok{),}
         \AttributeTok{is\_runner3 =} 
           \FunctionTok{as.numeric}\NormalTok{(run1\_dest\_id }\SpecialCharTok{==} \DecValTok{3} \SpecialCharTok{|}\NormalTok{ run2\_dest\_id }\SpecialCharTok{==} \DecValTok{3} \SpecialCharTok{|}
\NormalTok{                        run3\_dest\_id }\SpecialCharTok{==} \DecValTok{3} \SpecialCharTok{|}\NormalTok{ bat\_dest\_id }\SpecialCharTok{==} \DecValTok{3}\NormalTok{),}
         \AttributeTok{new\_outs =}\NormalTok{ outs\_ct }\SpecialCharTok{+}\NormalTok{ event\_outs\_ct,}
         \AttributeTok{new\_bases =} \FunctionTok{paste}\NormalTok{(is\_runner1,is\_runner2, is\_runner3, }\AttributeTok{sep =} \StringTok{""}\NormalTok{),}
         \AttributeTok{new\_state =} \FunctionTok{paste}\NormalTok{(new\_bases, new\_outs))}
\end{Highlighting}
\end{Shaded}

\newpage

\begin{Shaded}
\begin{Highlighting}[]
\NormalTok{retro2011 }\OtherTok{\textless{}{-}}\NormalTok{ retro2011 }\SpecialCharTok{|\textgreater{}}
  \FunctionTok{filter}\NormalTok{((state }\SpecialCharTok{!=}\NormalTok{ new\_state) }\SpecialCharTok{|}\NormalTok{ (runs\_scored }\SpecialCharTok{\textgreater{}} \DecValTok{0}\NormalTok{))}

\NormalTok{retro2011\_complete }\OtherTok{\textless{}{-}}\NormalTok{ retro2011 }\SpecialCharTok{|\textgreater{}}
  \FunctionTok{filter}\NormalTok{(outs\_inning }\SpecialCharTok{==} \DecValTok{3}\NormalTok{)}

\NormalTok{erm\_2011 }\OtherTok{\textless{}{-}}\NormalTok{ retro2011\_complete }\SpecialCharTok{|\textgreater{}}
  \FunctionTok{group\_by}\NormalTok{(bases, outs\_ct) }\SpecialCharTok{|\textgreater{}} \CommentTok{\# same as grouping by state}
  \FunctionTok{summarize}\NormalTok{(}\AttributeTok{mean\_run\_value =} \FunctionTok{mean}\NormalTok{(runs\_roi))}

\NormalTok{erm\_2011\_wide }\OtherTok{\textless{}{-}}\NormalTok{ erm\_2011 }\SpecialCharTok{|\textgreater{}} 
  \FunctionTok{pivot\_wider}\NormalTok{(}
    \AttributeTok{names\_from =}\NormalTok{ outs\_ct, }
    \AttributeTok{values\_from =}\NormalTok{ mean\_run\_value, }
    \AttributeTok{names\_prefix =} \StringTok{"Outs="}
\NormalTok{  )}

\FunctionTok{write\_csv}\NormalTok{(erm\_2011\_wide, }\StringTok{"erm\_2011\_wide.csv"}\NormalTok{)}

\NormalTok{erm\_2011\_wide }\SpecialCharTok{|\textgreater{}} \FunctionTok{kable}\NormalTok{()}
\end{Highlighting}
\end{Shaded}

\begin{longtable}[]{@{}lrrr@{}}
\toprule\noalign{}
bases & Outs=0 & Outs=1 & Outs=2 \\
\midrule\noalign{}
\endhead
\bottomrule\noalign{}
\endlastfoot
000 & 0.4711649 & 0.2546956 & 0.0971845 \\
001 & 1.4543568 & 0.9374359 & 0.3173913 \\
010 & 1.0582804 & 0.6501976 & 0.3091673 \\
011 & 1.9304897 & 1.3388641 & 0.5407407 \\
100 & 0.8350992 & 0.4960492 & 0.2179536 \\
101 & 1.7526998 & 1.1496169 & 0.4882597 \\
110 & 1.4144549 & 0.8739176 & 0.4222569 \\
111 & 2.1718266 & 1.4745146 & 0.7610094 \\
\end{longtable}

Now we use this expected run table to determine the value of every
event.

\begin{Shaded}
\begin{Highlighting}[]
\NormalTok{retro2011 }\OtherTok{\textless{}{-}}\NormalTok{ retro2011 }\SpecialCharTok{|\textgreater{}} 
  \FunctionTok{left\_join}\NormalTok{(erm\_2011, }\FunctionTok{join\_by}\NormalTok{(}\StringTok{"bases"}\NormalTok{, }\StringTok{"outs\_ct"}\NormalTok{)) }\SpecialCharTok{|\textgreater{}} 
  \FunctionTok{rename}\NormalTok{(}\AttributeTok{rv\_start =}\NormalTok{ mean\_run\_value) }\SpecialCharTok{|\textgreater{}} 
  \FunctionTok{left\_join}\NormalTok{(}
\NormalTok{    erm\_2011, }
    \FunctionTok{join\_by}\NormalTok{(new\_bases }\SpecialCharTok{==}\NormalTok{ bases, new\_outs }\SpecialCharTok{==}\NormalTok{ outs\_ct)}
\NormalTok{  ) }\SpecialCharTok{|\textgreater{}} 
  \FunctionTok{rename}\NormalTok{(}\AttributeTok{rv\_end =}\NormalTok{ mean\_run\_value) }\SpecialCharTok{|\textgreater{}} 
  \FunctionTok{replace\_na}\NormalTok{(}\FunctionTok{list}\NormalTok{(}\AttributeTok{rv\_end =} \DecValTok{0}\NormalTok{)) }\SpecialCharTok{|\textgreater{}} 
  \FunctionTok{mutate}\NormalTok{(}\AttributeTok{run\_value =}\NormalTok{ rv\_end }\SpecialCharTok{{-}}\NormalTok{ rv\_start }\SpecialCharTok{+}\NormalTok{ runs\_scored)}
\end{Highlighting}
\end{Shaded}

Note the code above also adds some very useful columns to the
play-by-play data related to the state and run expectancy at the start
and end of each play.

\begin{Shaded}
\begin{Highlighting}[]
\NormalTok{retro2011 }\SpecialCharTok{|\textgreater{}} 
 \FunctionTok{select}\NormalTok{(bat\_id, event\_cd, state, rv\_start,}
\NormalTok{        new\_state, rv\_end, runs\_scored, run\_value) }\SpecialCharTok{|\textgreater{}}
  \FunctionTok{head}\NormalTok{(}\DecValTok{10}\NormalTok{) }\SpecialCharTok{|\textgreater{}} 
  \FunctionTok{kable}\NormalTok{(}\AttributeTok{digits =} \DecValTok{3}\NormalTok{)}
\end{Highlighting}
\end{Shaded}

\begin{longtable}[]{@{}
  >{\raggedright\arraybackslash}p{(\linewidth - 14\tabcolsep) * \real{0.1250}}
  >{\raggedleft\arraybackslash}p{(\linewidth - 14\tabcolsep) * \real{0.1250}}
  >{\raggedright\arraybackslash}p{(\linewidth - 14\tabcolsep) * \real{0.0833}}
  >{\raggedleft\arraybackslash}p{(\linewidth - 14\tabcolsep) * \real{0.1250}}
  >{\raggedright\arraybackslash}p{(\linewidth - 14\tabcolsep) * \real{0.1389}}
  >{\raggedleft\arraybackslash}p{(\linewidth - 14\tabcolsep) * \real{0.0972}}
  >{\raggedleft\arraybackslash}p{(\linewidth - 14\tabcolsep) * \real{0.1667}}
  >{\raggedleft\arraybackslash}p{(\linewidth - 14\tabcolsep) * \real{0.1389}}@{}}
\toprule\noalign{}
\begin{minipage}[b]{\linewidth}\raggedright
bat\_id
\end{minipage} & \begin{minipage}[b]{\linewidth}\raggedleft
event\_cd
\end{minipage} & \begin{minipage}[b]{\linewidth}\raggedright
state
\end{minipage} & \begin{minipage}[b]{\linewidth}\raggedleft
rv\_start
\end{minipage} & \begin{minipage}[b]{\linewidth}\raggedright
new\_state
\end{minipage} & \begin{minipage}[b]{\linewidth}\raggedleft
rv\_end
\end{minipage} & \begin{minipage}[b]{\linewidth}\raggedleft
runs\_scored
\end{minipage} & \begin{minipage}[b]{\linewidth}\raggedleft
run\_value
\end{minipage} \\
\midrule\noalign{}
\endhead
\bottomrule\noalign{}
\endlastfoot
davir003 & 2 & 000 0 & 0.471 & 000 1 & 0.255 & 0 & -0.216 \\
nix-j001 & 2 & 000 1 & 0.255 & 000 2 & 0.097 & 0 & -0.158 \\
bautj002 & 3 & 000 2 & 0.097 & 000 3 & 0.000 & 0 & -0.097 \\
iztum001 & 14 & 000 0 & 0.471 & 100 0 & 0.835 & 0 & 0.364 \\
kendh001 & 3 & 100 0 & 0.835 & 100 1 & 0.496 & 0 & -0.339 \\
abreb001 & 6 & 100 1 & 0.496 & 000 2 & 0.097 & 0 & -0.399 \\
abreb001 & 14 & 000 2 & 0.097 & 100 2 & 0.218 & 0 & 0.121 \\
huntt001 & 20 & 100 2 & 0.218 & 001 2 & 0.317 & 1 & 1.099 \\
wellv001 & 2 & 001 2 & 0.317 & 001 3 & 0.000 & 0 & -0.317 \\
linda001 & 2 & 000 0 & 0.471 & 000 1 & 0.255 & 0 & -0.216 \\
\end{longtable}




\end{document}
